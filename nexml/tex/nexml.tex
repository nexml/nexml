\documentclass{article}
% $Id$

\author{Rutger A. Vos}
% This manuscript is supposed to be a manifesto of sorts, so anyone supporting
% the ideas outlined here is cordially invited to join in. Feel free to add
% your name here as another author: \author{Your Name Here}

\title{Nexml: an XML format for phylogenetic data}
\newcommand{\code}{\texttt}
\usepackage{fullpage}
\begin{document}
\maketitle
\tableofcontents

\section{Introduction}
Phylogenetic analyses are growing in size and scope.

Extensible markup language, or XML\cite{citeulike:2580222}, is an increasingly popular way to represent 
scientific data in general \cite{citeulike:1937046} and bioinformatic data \cite{citeulike:242} in particular. 
Here we present a new XML format for phylogenetic data, which in honor of the commonly used nexus \cite{citeulike:2011773} 
flat file format is called ``nexml''.

\section{RFC: nexml}
% In the first evoinfo\footnote{http://evoinfo.nescent.org} meeting the working group acknowledged the desirability of a formal 
% ontology as the basis for data interoperability, and has chosen the development of an ontology 
% as one of its goals. Such a formal ontology, for example in RDF, can be translated directly into a 
% serialization protocol and into a file standard definition such as an XML Schema. However, the requirements 
% of this ontology were not clearly established, and a serialization protocol for it would not automatically 
% result in a ``convenient'' next generation file standard. Hence, some attendants argued
% \footnote{https://www.nescent.org/wg/evoinfo/images/7/7a/Standards.ppt} that a more "bottom up" approach that 
% starts with the design of a new standard will help uncover the hidden requirements of the ontology and generate 
% the serialization protocol in the process. This page describes recent discussion and ongoing work on the 
% design of this new standard.
% 
% At the first meeting (20-23 May, 2007), the working group decided that work on the future data exchange 
% standard would primarily proceed under the auspices of Rutger Vos who has  
% reported on its progress at the Fall '07 meeting.
% 
% In internetworking and computer network engineering, Request for Comments (RFC) documents are a series of 
% memoranda encompassing new research, innovations, and methodologies applicable to Internet technologies. 
% RFCs are meant for peer review and to transmit new concepts, and are published through the Internet Society. 
% This document is therefore in strict terms not a ``real'' RFC, but it serves the same purpose: to provoke 
% comment and discussion, and to disseminate information about the design process of nexml. The status of 
% this document is neither that of a formal specification or recommendation, nor formally endorsed by evoinfo 
% in its current stage.

\subsection{Preamble}
The following subsections introduce the context within which nexml development takes place, describing 
the interoperability problems faced by phylogeneticists, the requirements nexml needs to meet to address 
these problems, and the development approach that is being taken to meet these requirements.

\subsubsection{Pressures}
Current practices and ongoing developments in phylogenetics add pressure to the need for a new data standard. 
The working group is compiling a list of these issues, 
some of which include:

\begin{itemize}

\item \textbf{Many flat file formats and dialects}: in recent years, phylogeneticists have adopted a number of different file formats 
to store data such as trees and sequence alignments. Unfortunately, most flat file formats used for this purpose generally lack 
an exhaustive, formal specification - and even if they do, they have been extended in various ways. Examples of this abounded 
during the meeting; for instance:

\begin{itemize}

\item pipe-delimited meta-data fields in fasta files

\item mutually exclusive syntax for polymorphy in nexus matrices: \verb={a,c}= vs. \verb=a&c=
 
\item ``hot comments'' \footnote{http://www.c2.com/cgi/wiki?HotComments} in tree descriptions
 
\item ``mixed'' data in mrbayes nexus matrices 

\item proprietary substitution model descriptions

\item extensions of molecular symbols list beyond IUPAC standard

\end{itemize}

\item \textbf{Data hard or impossible to validate automatically}: for many file formats, no validation procedure exists - and given the 
proliferation of mutually exclusive dialects it's difficult to see how such a procedure could deduce validity unambiguously. What 
might be valid ``nexus'' to one program, is rejected by another program. This makes many types of compatibility issues hard to solve 
without extensive knowledge of file conventions ``in the wild''.

\item \textbf{Bigger data sets}: meanwhile, phylogenetic analyses are continuously growing in size and complexity. Solving compatibility 
issues by hand is becoming too tedious.

\item \textbf{Increased automation}: the growth in size and complexity of analyses has promoted development of automated systems for data 
analysis (pipelines/workflows) and data storage (databases). The purpose of a workflow is obviously defeated if it requires manual 
editing of data files between steps in the analysis. As well, the reproducibility of results obtained from complex workflows is in 
doubt if no formal facility exists for logging the steps undertaken during an analysis. 

\item \textbf{More data types}: since the introduction of the nexus file format, new types of analyses (e.g. Bayesian, ML) have gained 
ground, which have introduced new types of information (substitution models, priors and posteriors on branches) for which no 
standardized format exists.

\item  \textbf{Nexus problems}: as the nexus file format is the most commonly used format in phylogenetics, the working group is compiling 
a separate list of nexus problems.

\end{itemize}

\subsubsection{Requirements}
To address the issues described in the preceding section, a new standard should meet at least the following requirements: 

\begin{itemize}

\item \textbf{Formalization}: the next standard implements a formal ontology and is
defined by a schema (as in phyloxml\footnote{http://www.phyloxml.org/}), such
that instance documents can be validated with great granularity, including for type safety for character data (per character), 
referential integrity between entities, validity of tree descriptions; 

\item \textbf{Extension}: the next standard extends current standards to include model
descriptions, reticulate trees, arbitrary attachments on nodes, trees, tree sets, sites, sequences, alignments, etc.; 

\item \textbf{Data integration}: the next standard allows for the integration of data from disparate sources, e.g. from multiple files, 
from databases, from sets of alignments and trees, from existing ontologies; 

\item \textbf{Legacy compatibility}: the next standard is organized conceptually, and can be serialized or otherwise represented in 
ways that are understandable for legacy software, such as being able to express the standard nexus subset;

\item \textbf{Analysis context description}: the next standard can represent meta-information such as analysis procedures, instance 
document changes, schema changes;

\item \textbf{Parseability}: the next standard is easily parsed without the need for custom tokenization, both in compact and verbose 
representations;

\end{itemize}

\subsubsection{Approach}

\begin{itemize}

\item Development of the data exchange standard is now ongoing with input from various contributors, including, in alphabetical order, Jason Caravas\footnote{https://www.nescent.org/wg\_phyloinformatics/PhyloSoC:Phylogenetic\_XML} Mark  Holder, David Maddison, Wayne Maddison, Peter Midford, Andrew Rambaut, Jeet Sukumaran and Rutger Vos as well as other members of the evoinfo group and attendants of the pPOD\footnote{http://www.phylodata.org} kickoff meeting.

\item The svn repository has migrated to sourceforge\footnote{https://nexml.svn.sourceforge.net/svnroot/nexml}. It contains a modularized XML schema (discussed on powerpoint slides\footnote{https://www.nescent.org/wg/evoinfo/images/2/23/Nexml.ppt}) and various instance documents\footnote{https://nexml.svn.sourceforge.net/svnroot/nexml/examples/} demonstrating representations of common phylogenetic data.
	
\item Nexml parsers in perl\footnote{https://nexml.svn.sourceforge.net/svnroot/nexml/perl/}, python and in java\footnote{https://nexml.svn.sourceforge.net/svnroot/nexml/java/} are under development.

\item The domain name http://www.nexml.org has been registered, which will serve to identify the nexml namespace and for which some sort of web presence (e.g. a public storage of schema files, information about the format, an online validator) are in the early planning stages.	

\item NESCent has provided a staging server\footnote{http://nexml-dev.nescent.org} from which this web presence will be launched on the production server. At present we use this domain to aggregate data about the development process in one location.	

\item To facilitate development of the ontology, concepts introduced by the design of the schema and discussed on this wiki page will crossreference with the glossary.	

\item An online validation service\footnote{http://www.nexml.org/nexml/validator} is under development. You can reuse it by placing this in an html document: 

\begin{verbatim}
<form 
  action="http://www.nexml.org/nexml/validator" 
  enctype="multipart/form-data" 
  method="post">
    <input type="file" name="file"/>
    <input type="submit"/>
</form>
\end{verbatim}
	
Or use it as a simple REST service: HTTP response code 201 means it's valid, 400 means it isn't. The validator goes through a two-step process:

\item \textbf{grammar-based validation}: this is the part where we check an input file for sensible placement of elements, attributes and text nodes. We do this by validating against the xml schema using a Xerces-J validator writting by Terri Schwartz of the CIPRES-SDSC team. Unfortunately, some constraints we like to place on files cannot be expressed in schema language. For this we have to resort to the second step.	

\item \textbf{rule-based validation}: here we test whether chained remote references (rules such as ``matrix row X is part of characters block Y, which is linked to OTUs block Z. Therefore, X can only refer to an OTU that is part of Z'') make sense. We do this by parsing the file using Bio::Phylo's nexml parser. 

\end{itemize}

\subsection{Design}
The following subsections describe the design of the developing nexml standard.
These descriptions are likely to both be in flux, and lagging behind the formal (one, true) XSD description in the svn repository\footnote{http://nexml.svn.sourceforge.net/viewvc/nexml/xsd/}. The reader is therefore advised to follow the links in the subsections to the XSD fragments and XML instance document examples that define the elements.

\subsubsection{General design}

\subsubsection{Verbosity}
XML in general is usually far more verbose than flat file formats that convey the same information. Nexml is designed to allow both for verbose and compact representation of data. The rationale is that some use cases (e.g. submission of morphological observations to a database) require extensive annotation, whereas others (e.g. submission to a processing engine as part of a workflow) don't require a lot of metadata, only the bare bones needed to complete the analysis step.

\subsubsection{References}
Elements other than xml structural placeholders (e.g. the \code{matrix} element
that lumps matrix rows together) all have the following properties:
% \begin{itemize}
% \item a required \code{id} attribute. This attribute is not a standard XML id (which are file scoped), but of type \code{xs:NCName}, which have the same string format restrictions but are free to have their scope defined by the schema. In the nexml schema, all such ids are scoped within their enclosing elements. This means that nodes must have an id that is unique within the tree that contains them, matrix rows must have ids unique within the matrix, etc. The rationale is that this minimizes the risk of clashes when combining data from multiple sources, and makes it easier to solve clashes should they occur (by re-assigning the id of the enclosing element).
% \item some elements must refer to other elements: e.g. a matrix \code{row} element must refer to a \code{otu} element. By convention, the attribute used to define the id of another element is named after that referenced element. For example, \code{<row otu="t1" id="r1"/>} means that the row element identified by id \code{r1} links to a otu element whose id is \code{t1}. Referenced elements in nexml always precede their references, which is why \code{otus} elements come before \code{characters} elements, for example.
% \end{itemize} 

\subsubsection{Metadata}
% \begin{itemize}
% To allow for the preservation of arbitrary labels and annotations, nexml elements have:
% \item an optional \code{label} attribute, which is used to specify a human-readable label for the element. The only restriction on this attribute's value is that
% it is a valid xml string, which means that reserved xml tokens (such as the greater-than symbol >) must be escaped using xml entities. 
% \item zero or more \code{<dict/>} elements (dictionary attachments) can be
% enclosed within them. These attachments, if present, precede any other enclosed elements (the rationale is similar to that for having referenced elements precede their references: it facilitates stream based parsing strategies).
% \end{itemize}

\subsubsection{Other common attributes}
In addition to the shared attributes and structures described above, nexml elements also optionally have:

\begin{itemize}

\item a \code{class} attribute. This attribute has an id reference to a class
defined earlier as its value. Its function is analogous to that of sets (e.g. OTU sets, character sets) in nexus, but instead of the set 
declaring what it contains here the nexml elements declare what set(s) they belong to. This is the idiomatic 
approach for XML, and is compatible with CSS styling.

\item an \code{xml:base} attribute, from the core XML
namespace\footnote{http://en.wikipedia.org/wiki/XML\_Namespace}. 
This attribute has an absolute URL as value, which is the base address upon which relative links defined later 
in a document are built. There is a similar facility in HTML (defined in the head of a document), which in 
practice is used to shorten URLs in pages by defining their shared path fragments in one location. The rationale 
for inclusion in nexml is similar, with the additional consideration that some 
XInclude\footnote{http://en.wikipedia.org/wiki/XInclude} processors place the
attribute by default in the post-include
infoset\footnote{http://en.wikipedia.org/wiki/Infoset}. 

\item an \code{xlink:href} attribute, from the
XLink\footnote{http://en.wikipedia.org/wiki/XLink} namespace. This attribute has an absolute or relative URL as 
its value, which is used to create a unidirectional link from the 
element that has the attribute to another location.

\item an \code{xml:id} attribute, from the core XML namespace. This attribute
has a string of the same format as the nexml id as its value, except this id is file-scope unique. The attribute has no defined function within 
nexml, but some XInclude processors might include it.

\item an \code{xml:lang} attribute, from the core XML namespace. This
attribute has an RFC3066\footnote{http://www.ietf.org/rfc/rfc3066.txt} compliant language code
(e.g. \code{en-US}) as its value, which is used to indicate the natural
language in which text of the element or its children are written.

\end{itemize}

\subsection{Element description}

\subsubsection{Root}
% [[http://nexml.svn.sourceforge.net/viewvc/*checkout*/nexml/xsd/nexml.xsd schema]], 
% [[http://nexml.svn.sourceforge.net/viewvc/*checkout*/nexml/examples/nexml.xml example]]

(Like all standalone xml files, nexml files start with the \code{<?xml version="1.0" encoding="some character encoding scheme"?>}
processing instruction. The encoding scheme is used to specify what character set - e.g. some unicode flavor - is used. Hence, nexml files can include OTU descriptions in simplified Chinese, 
for example, without confusing parsers. Naturally,
i18n\footnote{http://en.wikipedia.org/wiki/Internationalization\_and\_localization}
in terms of character sets is best used in combination with
\code{xml:lang} attributes where appropriate.)

The root element of the schema is called \code{nexml}. This root element
has the following attributes:

\begin{itemize}

\item a required \code{version} attribute whose value is a decimal number indicating the nexml schema version. Until a revision
occurs after the first release of nexml the value must be 1.0.

\item an optional \code{generator} attribute, which is used to identify the program that generated the file. The attribute's value
is a free form string.
In addition, it will commonly define a number of xml namespace prefixes. (Where it says ``by convention'' in the list below, the convention applies to the three-letter prefixes which are free to vary in most cases, not the namespaces themselves):

\item the xml namespace prefix that identifies xml schema semantics that might be inlined in the file. By convention this is of the
format \\ \code{xmlns:xsi="http://www.w3.org/2001/XMLSchema-instance"} so that parts of schema language used inside nexml
(e.g. where a concrete subclass must be specified) are identified by the \code{xsi} prefix.

\item the nexml namespace prefix, by convention of the format \\ \code{xmlns:nex="http://www.nexml.org/1.0"}, so that  locations
where nexml specific types are referenced (e.g. data type subclasses) these are identified by their \code{nex} prefix.

\item the xml namespace prefix, required to be of the format \\ \code{xmlns:xml="http://www.w3.org/XML/1998/namespace"}. This may
be used, for example, to specify the base address of imported nexml snippets (using the \\
\code{xml:base="http://example.com/base.xml"} attribute) and to specify the language in which certain element contents are
written (using, say, \code{xml:lang="nl"}).

\end{itemize}

Lastly, to associate the instance document with the nexml schema, it requires an attribute to specify the nexml schema location, and
the namespace it applies to. This is of the format \\ \code{xsi:schemaLocation="http://www.nexml.org/1.0 nexml.xsd"} (in a stable
release the location of the schema would not be a local path - such as nexml.xsd - but a url). Notice that this attribute is a schema
language snippet (identified by the \code{xsi:} prefix) that identifies a namespace (\code{http://www.nexml.org/1.0})
and associates it with a physical schema location (\code{../nexml.xsd}).


Together, this makes the root element look something like the following:
\begin{verbatim}
 <?xml version="1.0" encoding="ISO-8859-1"?>
 <nex:nexml 
   version="1.0" 
   generator="eclipse"
   xmlns:xsi="http://www.w3.org/2001/XMLSchema-instance"
   xmlns:xml="http://www.w3.org/XML/1998/namespace"
   xmlns:nex="http://www.nexml.org/1.0"
   xsi:schemaLocation="http://www.nexml.org/1.0 ../nexml.xsd">
   <!-- contents go here -->
 </nexml>
\end{verbatim}
The root element can hold:

\begin{itemize}

\item zero or more attachment dictionaries, 

\item one or more OTUs elements, 

\item zero or more characters elements, 

\item zero or more trees elements (in mixed order with characters elements),

\end{itemize}

\subsubsection{OTUs}
% [[http://nexml.svn.sourceforge.net/viewvc/*checkout*/nexml/xsd/taxa/taxa.xsd schema]], 
% [[http://nexml.svn.sourceforge.net/viewvc/*checkout*/nexml/examples/taxa.xml example]],
% [[http://www.nescent.org/wg\_evoinfo/ConceptGlossary#Operational\_Taxonomic\_Unit\_\_.28OTU.29 term]],

In some phylogenetic analyses (such as Bayesian analyses that yield a set of trees) many nodes 
\footnote{http://www.nescent.org/wg\_evoinfo/ConceptGlossary\#Phylo\_Node} - in different trees 
\footnote{http://www.nescent.org/wg\_evoinfo/ConceptGlossary\#Phylogeny} - refer to the same sequence. On the other hand, in some 
other analyses (such as those that involve simulation of a set of sequence alignments) many sequences - in different alignments - 
refer to the same node in the generating tree. This relationship can be normalized by creating a third entity from which one-to-many 
links point both to nodes and sequences. In nexus files, these entities are defined in the taxa block using a set of labels that 
sequences and nodes later on in the file must refer to. In nexml, this same functionality is provided by \code{otus} elements. 
The name change is a result of the progressing integration between phylogenetics and taxonomy 
\footnote{http://www.nescent.org/wg\_evoinfo/ConceptGlossary\#Organismal\_Taxonomy}, which now causes concept confusion because of the 
overloading of the term ``taxa'' when it was introduced in the nexus standard.

In its simplest form, a \code{otus} element looks something like this:
\begin{verbatim}
<otus id="tax1">
  <otu id="t1"/>
  <otu id="t2"/>
</otus>
\end{verbatim}
That is, the \code{otus} element and its contained \code{otu} elements require a (file level unique) id attribute. In 
addition, these elements can have an optional \code{label} attribute that defines a human readable name for the element, and 
can contain dictionary attachments.

\subsubsection{Characters}
% [[http://nexml.svn.sourceforge.net/viewvc/*checkout*/nexml/xsd/characters/characters.xsd schema]], 
% [[http://nexml.svn.sourceforge.net/viewvc/*checkout*/nexml/examples/characters.xml example]],
% [[http://www.nescent.org/wg_evoinfo/ConceptGlossary#Character-State_Data_Matrix term]]

The \code{characters} element is somewhat analogous to the nexus characters blocks: it stores data such as molecular sequences, 
categorical data or continuous data. The element is different from the nexus characters block in that it allows for more detailed 
specification of the allowed states \textit{per character}, strict validation of the observed states, annotation of characters (columns), 
states, rows and individual observations using dictionaries. In addition, the \code{characters} element is 
designed to allow for representation of non-homologized data: the element is more accurately described as a bucket of observations and 
the allowed parameter space for those observations. Only if the boolean attribute \code{aligned} of the \code{matrix} 
element is set to ``1'' (true), can subsequent observations be assumed to be homologous across \code{row} elements.

The schema specifies the \code{characters} element to be of an abstract type, so that instance documents need to specify the 
concrete subclass (i.e. datatype) using the \code{xsi:type} attribute. At present, six data types are supported: DNA, protein, 
restriction sites, standard categorical, continuous and RNA. For each of these data types there are two subclasses, whose names are 
constructed as "data type"Seqs (e.g. DnaSeqs) and "data type"Cells (e.g. DnaCells), the former being a compact representation with 
states listed as tokens in a sequence, the latter a verbose representation with states marked up granularly.

\subsubsection{DNA}
In its most compact form, a DNA sequence alignment as expressed in a \code{characters} element would look something like this:
\begin{verbatim}
<characters otus="tax1" id="m1" xsi:type="nex:DnaSeqs">
  <matrix aligned="1">
    <row id="r1" otu="t1"><seq>AACATATCTC</seq></row>
    <row id="r2" otu="t2"><seq>ATACCAGCAT</seq></row>
    <row id="r3" otu="t3"><seq>GAGGGTATGG</seq></row>
    <row id="r4" otu="t4"><seq>GGTCTTAGAG</seq></row>
    <row id="r5" otu="t5"><seq>CGTCACAGTG</seq></row> 
   </matrix>		
 </characters>
\end{verbatim}
\code{characters} elements are polymorphic in two ways:

\begin{enumerate}

\item The \code{xsi:type} attribute defines the data type, and thus the concrete character matrix subclass. In the example, the 
DNA subclass restricts the allowed symbols in the matrix to the IUPAC single character nucleotide symbols, allows omission of 
per-character state definitions, and allows concatenation of symbols into a string inside the \code{seq} element.

\item The second polymorphism is in the granularity of observation mark up. The example above shows a compact representation where 
characters are concatenated as a string.

\end{enumerate} 

The RNA subclasses are virtually identical to the equivalent DNA subclasses (only T vs U). The protein and restriction data types 
are similar in that they also don't require character definitions and can also be expressed as concatenated strings. This is because 
they both use single character symbols, whose allowed symbols are well-defined, namely as the IUPAC single character amino acid 
symbols and as 0 or 1, respectively. In all compact representations, gaps \footnote{http://www.nescent.org/wg\_evoinfo/ConceptGlossary\#Gap} 
and missing data are encoded as separate alignments, so that multiple alignments can be associated with the same sequence.

\subsubsection{STANDARD}
For other data types, the behavior of the \code{characters} element is different. For categorical data 
(\code{xsi:type="nex:StandardCells"}), the \code{row} elements inside the \code{matrix} element must be preceded 
by a \code{format} element that specifies the allowed states 
\footnote{http://www.nescent.org/wg\_evoinfo/ConceptGlossary\#Character-State} per character 
\footnote{http://www.nescent.org/wg\_evoinfo/ConceptGlossary\#Character}. An example is shown below:
\begin{verbatim}
<!-- this goes inside matrix element -->
<!--
  Because categorical characters in instances of this subclass
  don't have a priori known numbers of states, they must be
  defined using the 'format' element. 
-->
<format>
  <!--
    The first elements inside a format element are stateset
    definitions. In this example, there is a set of four
    states, each tagged with an id. The symbol attribute is
    a shorthand token. 
  -->
  <states id="states1">
    <state id="s1" symbol="1"/>                
    <state id="s2" symbol="2"/>                
    <state id="s3" symbol="3"/>                
    <state id="s4" symbol="4"/>  
    <mapping state="s1" mstaxa="polymorphism"/>
    <mapping state="s2" mstaxa="polymorphism"/>              
  </states>
  <!--
    The matrix in this example contains two columns, both
    referring to the same stateset - and so cells in both
    columns can occupy one of four states. 
  -->
  <char states="states1" id="c1"/>
  <char states="states1" id="c2"/>
</format>
<!-- row elements follow -->
\end{verbatim}

In this case, then, the matrix holds two four-state characters. State \code{s4} functions as an ambiguity code, a state that 
actually means a set of two other states (\code{s1} and \code{s2}). Because the \code{mstaxa} attribute is set to 
\code{polymorphism}, the mapping to the two other states indicates true polymorphism (i.e. both states are observed in a population). The other value for the \code{mstaxa} attribute is \code{uncertain}. In practice, "polymorphism" can be read as "AND", and "uncertain" as "OR". \code{states}, \code{state} and \code{char}  elements can take dictionary attachments, which are loosely analogous to - but more powerful than - the nexus tokens CHARSTATELABELS and CHARLABELS, respectively.

Subsequent \code{row} elements contain the mappings between the defined states and the actual observations. For example:
\begin{verbatim}
<!-- preceded by definitions, inside matrix element -->
<row id="r1" otu="t1">
  <!--
    Each cell must contain a reference to the column 
    it belongs to, and to a state allowed within that
    column. 
  -->
  <cell char="c1" state="s2"/>
  <cell char="c2" state="s2"/>
</row>
\end{verbatim}

This structure means that the entity defined by OTU ``t2'' has state ``s2'' (symbol 2) for both characters. Therefore, \code{cell} 
elements within different \code{row} elements are homologous\footnote{http://www.nescent.org/wg\_evoinfo/ConceptGlossary\#Homology} 
if they have the same value for the \code{char} attribute.
DNA, RNA, protein and restriction data can be described in a similar, verbose way, with the ``state'' attribute's value being an IUPAC 
nucleotide symbol, an IUPAC amino acid symbol or a boolean (0/1), respectively, and the ``char'' attribute's value being the (zero-based) 
column number.

In a compact representation, the same STANDARD information is marked up like this:
\begin{verbatim}
 <row id="r2" otu="t2"><seq>2 2</seq></row>
\end{verbatim}

Notice how this is similar to the compact DNA representation, except in the allowed symbols that are used (integers instead of IUPAC 
symbols) and that the symbols are space-separated (this is because STANDARD states aren't necessarily single-character symbols: 
integers greater than 9 are allowed also).

\subsubsection{CONTINUOUS}
For continuous data, the \code{format} element defines the characters (i.e. \code{char} elements) but not their states, 
and observations values (i.e. either the \code{state} attribute in verbose notation, or space-separated symbols in compact 
notation) are arbitrary precision floating point numbers.

\subsubsection{Combining data}
In some analyses, data of different types is analyzed jointly (e.g. mrbayes does this). Nexml does not currently facilitate this. It 
is likely that this will be implemented either using matrix sets or (less likely) using a MIXED concrete subclass.

\subsubsection{Trees}
% [[http://nexml.svn.sourceforge.net/viewvc/*checkout*/nexml/xsd/trees/trees.xsd schema]], 
% [[http://nexml.svn.sourceforge.net/viewvc/*checkout*/nexml/examples/trees.xml example]]

Due to their nesting, tree descriptions as nested elements (as suggested in Joe Felsenstein's book \textit{Inferring Phylogenies}) 
can pose special problems for xml parsers: a parser can only hand off an element once all its children have been processed and stored 
in memory. Large trees described using nested elements can therefore develop huge memory requirements. Hence, nexml describes trees as 
node and edge\footnote{http://www.nescent.org/wg\_evoinfo/ConceptGlossary\#Branch\_.28Edge.29} tables instead, following the semantics 
for GraphML\footnote{http://graphml.graphdrawing.org/} (which is also discussed separately in the context of the study of related artefacts).
\subsubsection{Tree}
% [[http://www.nescent.org/wg_evoinfo/ConceptGlossary#Phylogeny term]]

The concrete subclasses \code{IntTree} and \code{FloatTree} describe a tree shape following GraphML semantics. The 
classes differ in that the optional \code{length} attribute is either an integer or a IEEE 754-1985 compliant floating 
point number. Below is an example:
\begin{verbatim}
<!-- A tree with float edges. -->
<tree id="tree1" xsi:type="nex:FloatTree" label="tree1">
  <node id="n1" label="n1" root="true"/>
  <node id="n2" label="n2" otu="t1"/>
  <node id="n3" label="n3"/>
  <node id="n4" label="n4"/>
  <node id="n5" label="n5" otu="t3"/>
  <node id="n6" label="n6" otu="t2"/>
  <node id="n7" label="n7"/>
  <node id="n8" label="n8" otu="t5"/>
  <node id="n9" label="n9" otu="t4"/>
  <edge source="n1" target="n3" id="e1" length="0.34534"/>			
  <edge source="n1" target="n2" id="e2" length="0.4353"/>
  <edge source="n3" target="n4" id="e3" length="0.324"/>
  <edge source="n3" target="n7" id="e4" length="0.3247"/>
  <edge source="n4" target="n5" id="e5" length="0.234"/>
  <edge source="n4" target="n6" id="e6" length="0.3243"/>
  <edge source="n7" target="n8" id="e7" length="0.32443"/>
  <edge source="n7" target="n9" id="e8" length="0.2342"/>
</tree>
\end{verbatim}
This is an XML representation of the newick string \code{(((t4,t5)n7,(t2,t3)n4)n3,t1)n1;}. In this representation, the 
root \footnote{http://www.nescent.org/wg\_evoinfo/ConceptGlossary\#Root} is principally identified by having in-degree of zero, i.e. 
no \code{edge} element exists with a \code{target} attribute that references that node. However, an additional 
\code{root} attribute is also used to further indicate that this tree is in fact considered truly rooted. This attribute 
may be used on multiple nodes, to indicate multiple rootings. Tips 
\footnote{http://www.nescent.org/wg\_evoinfo/ConceptGlossary\#Terminal\_Node} are identified by there being no \code{edge} elements 
with \code{source} attributes that reference them. 

To add additional objects to nodes or edges, such as bootstrap 
\footnote{http://www.nescent.org/wg\_evoinfo/ConceptGlossary\#Bootstrap\_.28Bootstrap\_confidence\_value.3B\_Bootstrap\_support\_value.29} 
values, a dictionary attachment is used.

\subsubsection{Network}
% [[http://www.nescent.org/wg_evoinfo/ConceptGlossary#Phylogenetic_Network term]]

The \code{IntNetwork} and \code{FloatNetwork} subclasses only differ from the tree subclasses in that the key constraints on the in-degree of nodes is lessened, so that a node can have multiple parents. In the example below, node \code{n6} has an additional parent node \code{n7}, creating a reticulation:
\begin{verbatim}
<!-- A network with int edges. -->
<tree id="tree3" xsi:type="nex:IntNetwork" label="tree2">
  <node id="n1" label="n1"/>
  <node id="n2" label="n2" otu="t1"/>
  <node id="n3" label="n3"/>
  <node id="n4" label="n4"/>
  <node id="n5" label="n5" otu="t3"/>
  <node id="n6" label="n6" otu="t2"/>
  <node id="n7" label="n7"/>
  <node id="n8" label="n8" otu="t5"/>
  <node id="n9" label="n9" otu="t4"/>
  <edge source="n1" target="n3" id="e1" length="1"/>			
  <edge source="n1" target="n2" id="e2" length="2"/>
  <edge source="n3" target="n4" id="e3" length="3"/>
  <edge source="n3" target="n7" id="e4" length="1"/>
  <edge source="n4" target="n5" id="e5" length="2"/>
  <edge source="n4" target="n6" id="e6" length="1"/>
  <!-- extra edge -->
  <edge source="n7" target="n6" id="e9" length="1"/> 
  <edge source="n7" target="n8" id="e7" length="1"/>
  <edge source="n7" target="n9" id="e8" length="1"/>
</tree>
\end{verbatim}

\subsubsection{Sets}
% [[http://nexml.svn.sourceforge.net/viewvc/*checkout*/nexml/xsd/meta/sets.xsd schema]]

No formal design for a "set" of entities (a node set, an OTU set, etc.) exists yet. The intention is that nexml will provide a 
general facility to identifiable group elements, so that annotations and substitution models can be attached to these groups or 
to indicate that sets (e.g. matrices with different data types) are to be analyzed jointly. A first step has been the addition of 
the optional \code{class} attribute on elements. This attribute has a vector of identifiers as its value, such that the element 
can specify to which classes (i.e. sets) it belongs. This is different from nexus, where sets specify which elements they contain, 
but the approach suggested here is more idiomatic for xml, and is compatible with styling (e.g. using CSS).

\subsubsection{Dictionaries}
% [[http://nexml.svn.sourceforge.net/viewvc/*checkout*/nexml/xsd/meta/annotations.xsd schema]]

Future developments in phylogenetics are impossible to predict - new support values for nodes on trees may emerge, 
new types of annotations for DNA sequences, etc. In addition, the amount and types of metadata that can be attached to 
phylogenetic data are unlimited: some researchers may want to attach taxonomy database ids to otu elements, or genbank 
accession numbers to DNA and so on. A data exchange standard that attempts to limit the universe of ``stuff'' to attach to 
``things'' is likely to be headed for immediate obsolescence. The nexml standard therefore allows for attachment of arbitrary 
key/value pairs to: 

\begin{itemize}

\item the root element, 

\item otus and otu elements,

\item the characters element and its character definitions, state definitions, matrix rows, sequences and individual 
observations,

\item the trees, individual tree elements and nodes.

\end{itemize}

Or, in general, any element that inherits from the [http://nexml07gsoc.googlecode.com/svn/trunk/xsd/abstract.xsd Annotated] abstract 
type (i.e. everything other than placeholder elements).

The semantics of these key/value pairs follow the conventions used in Apple's Mac OS X property list
\footnote{http://en.wikipedia.org/wiki/Plist\#Mac\_OS\_X} format, a simple example being:
\begin{verbatim}
<dict>
  <key>description</key>
  <string>This is a string based description of an element</string>
</dict>
\end{verbatim}
A \code{dict} element contains a sequence of key/value pairs, where the \code{key} element contains a string 
(scoped to be unique within the enclosing dictionary), and the value can take on a number of different types:

\begin{itemize}

\item \code{integer} or \code{integervector}

\item \code{float} or \code{floatvector}

\item \code{string} or \code{stringvector}

\item \code{boolean} or \code{booleanvector}

\item \code{id} or \code{idvector}

\item \code{date} or \code{datevector}, i.e. a restricted date/time format

\item \code{url} or \code{urlvector}

\item \code{base64} or \code{base64vector}, i.e. for base64 encoded binary data such as images;

\item \code{dict} or \code{dictvector}, i.e. another dictionary, yielding a recursive data structure;

\item \code{any}, i.e. XML from another namespace such as XHTML (for marked up text), SVG (for graphs), ant build scripts, etc.

\end{itemize}

The implication is that dictionaries map onto data structures much like hashtables, python dictionaries, or perl hashes but with 
type-safe values. In principle, this structure thus allows all kinds of things to be attached to elements, the only downside 
being that different researchers might use different substructures or types. For example:
\begin{verbatim}
<dict><key>genbank</key><id>NM_052988.2</id></dict>
\end{verbatim}
and:
\begin{verbatim}
<dict><key>accession</key><id>NM_052988.2</id></dict>
\end{verbatim}
would not be recognized to mean the same thing without human intervention. The solution is to create restricted subclasses such as 
an ``accession'' class:
\begin{verbatim}
<dict xsi:type="db:accession"><key>genbank</key><id>NM_052988.2</id></dict>
\end{verbatim}
which would define a limited set of keys (e.g. genbank, dbj, ebi) and what type of value (e.g. \code{id}) is to follow. The 
same approach can obviously be used to refer to other databases such as those for taxonomies (e.g. ITIS) or morphology databases 
(e.g. Morphbank). In addition, \code{xsi:type} subclasses can be created to restrict other common types of attachments such 
a branch lengths or various node support values.

Extension of the nexml standard by this kind of subclassing does not require changes in the core schema (which only references the 
superclass) so should not confuse (well-designed) nexml parsers.

\subsubsection{Core restricted dictionaries}
One of the predefined\footnote{http://nexml07gsoc.googlecode.com/svn/trunk/xsd/meta/meta.xsd} optional dictionaries is a 
restricted structure that can be used to store revision control system (i.e. cvs, svn) keywords in a standardized format. Example:
\begin{verbatim}
<dict xsi:type="nex:RCSDate">
  <key>date</key>
  <string>$Date: 2007-07-30 18:36:51 +0200 (Mon, 30 Jul 2007) $</string>
</dict>
<dict xsi:type="nex:RCSRev">
  <key>rev</key>
  <string>$Rev: 74 $</string>
</dict>
<dict xsi:type="nex:RCSAuthor">
  <key>author</key>
  <string>$Author: rutgeraldo $</string>
</dict>
<dict xsi:type="nex:RCSURL">
  <key>url</key>
  <string>$URL: https://nexml07gsoc.googlecode.com/svn/trunk/examples/verbose.xml $</string>
</dict>
<dict xsi:type="nex:RCSId">
  <key>id</key>
  <string>$Id$</string>
</dict>
<dict xsi:type="nex:RCSHeader">
  <key>header</key>
  <string>$Header: $</string>
</dict>
\end{verbatim}

\subsubsection{3rd party restricted dictionaries}

The repository shows an example of how software authors can introduce their own restricted dictionaries in a separate namespace. There's 
an example file\footnote{http://nexml07gsoc.googlecode.com/svn/trunk/examples/mesquite.xml}, and here's an 
example schema\footnote{http://nexml07gsoc.googlecode.com/svn/trunk/examples/mesquite.xsd} that illustrates this.

\section{Resources}

\subsection{XML Parsers}
Below are the most commonly used open source XML parsers. Which should I use? Apache and MIT licenses are both compatible with 
closed source projects, i.e. no copyleft requirement. SAX is the stream-based api (more memory-efficient, but need to keep track 
of context yourself), DOM is the tree-based api (easier to traverse, harder on memory). Expat is also used under the hood by perl 
xml parser XML::Twig. It's (supposed to be) the fastest xml parser. Gnome is sometimes hard to build on Windows, so I'd use expat 
for SAX or xerces for DOM. For java, the nexml class libraries use whatever lives under org.xml.sax.*

\begin{itemize}

\item http://xerces.apache.org is Apache's XML parser, implements the SAX (recommended), SAX2 and DOM apis in C++, Java and Perl. 
Released under Apache license.

\item http://xmlsoft.org is Gnome's XML parser, implements SAX/SAX2/DOM, written in C. Released under MIT License.

\item http://expat.sourceforge.net implements the SAX api in C. Released under Apache license.

\end{itemize}

\section{Other XML projects}
http://hobit.gsf.de/wiki/display/wiki/XML+Schemas maintains a wiki of other bioinformatics xml projects.

\section{Outstanding issues, not yet in RFC}

\subsection{Evoinfo defragmentation}
The schema discussed here is to be an implementation of the ontology under development by the working group. 
Once formalization of the ontology begins (e.g. in OWL), the nexml project and the OWL development will keep a close eye on each 
other. In addition, the phylogenetic terms used in this page are to be compatible with those in the glossary.

\subsection{General design issues}
To meet the requirements for analysis context description, the nexml standard needs to accommodate more metadata 
about the contents of an entire nexml file:

\begin{itemize}

\item \textbf{Contents processing}: conceivably, a nexml file shouldn't just log how it has changed, but also how it should be processed - 
i.e. it might specify the steps involved in a workflow, settings used, etc. If a workflow solely consists of command line programs, 
it's possible these can be chained together using ant, for which the commands could be embedded as xml in a nexml file.

\end{itemize}

\subsection{Higher-level structure}
To meet the requirements for data integration, a higher-level ``project'' structure that can include multiple nexml files may be 
necessary. A facility called XInclude exists that allows for merging of xml files into a common Infoset. Unfortunately, this may 
create id clashes between included files. Some form of processing before inclusion may be necessary, e.g. as described 
elsewhere\footnote{http://www.xml.com/pub/a/2007/03/28/xinclude-processing-in-xslt-with-xipr.html}.

\subsection{Element changes}

\subsubsection{Characters}
Some data type subclasses need a way to indicate reading frames.

In addition, the characters element needs to be able to provide alternative alignments, possibly using 
a restricted dictionary that defines, per matrix row, the mapping between location in the unaligned sequence and 
homologized position, probably using an \code{integervector} value type. An attractive suggestion (from MTH) is to attach a 
vector of integers where the value of each integer is an index in the unaligned sequence, and the number of occurrences of that value 
is the number of gaps to insert. For example, \code{0 0 0} means that three gap symbols are prepended to the sequence.

Lastly, characters from different data types might be analyzed jointly. This might be implemented using the sets element.

\subsubsection{Distances}
Nexml needs some notion of pairwise distances\footnote{http://www.nescent.org/wg\_evoinfo/ConceptGlossary\#Distance\_Matrix} between OTUs.

\subsubsection{Splits}
Nexml needs some notion of OTU bipartitions \footnote{http://www.nescent.org/wg\_evoinfo/ConceptGlossary\#Bipartition\_.28Split.29} / splits.

\subsubsection{Trees}
Compact representations of trees (e.g. compressing of sets of trees by reusing subtrees, representing trees as ancestor functions in integer vectors) may be implemented. There may be a need, under some use cases, for a branch leading up to the root, so this needs to be implemented somehow.

\subsubsection{Sets}
Some way of grouping elements of the same kind needs to be implemented, e.g. node sets, tree sets, and so on.

\subsubsection{Substitution models}
The nexml standard needs to be able to express substitution models, and which characters/trees they apply to. Given that 
the work to date on this project has been implemented using IDL, there's some chance that 
this might be transformed\footnote{http://search.cpan.org/~perrad/CORBA-XMLSchemas-0.41/idl2xsd.pl} directly into XML schema, 
then added to nexml. During a work meeting in Lawrence, KS, some contributors sketched out an 
example\footnote{http://nexml07gsoc.googlecode.com/svn/trunk/examples/models.xml} of what a model description in XML might look like.

\section{References}
\bibliographystyle{alpha}
\bibliography{refs}

\end{document}